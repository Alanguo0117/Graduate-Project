\documentclass[12pt]{article}
\usepackage{setspace}
\usepackage[margin = 1.4in]{geometry}
\usepackage{amsfonts, amsmath, amssymb}
\usepackage[none]{hyphenat}

\begin{document}
\begin{titlepage}
   \vspace*{\stretch{1.0}}
   \begin{center}
      \Large\textbf{Genetic Alogrithm for CNN Hyperparametr Optimization}\\
      \vspace*{\stretch{0.3}}
      \large\textit{Tianrui Guo}\\
      \vspace*{\stretch{0.1}}
      \large Department of Applied Mathematics\\
      \vspace*{\stretch{0.1}}
      \large\textit{April 20, 2019}
   \end{center}
   \vspace*{\stretch{2.0}}
\end{titlepage}

\pagebreak
\setlength{\baselineskip}{10mm}
\begin{center}
\LARGE \textbf{Abstract}
\end{center}
In this big data era, we developed many efficient and effective ways to analyze our database. Among various techniques, computer vision is a fancy and important one. One of the most significant technique in modern computer vision is convolutional neural network which is a typical method in deep learning. When we process image classification or image recognition, we inevitably encounter a problem of determine the hyperparamters such as the number of Conv channels, the size of filters or the dimension of the dense layers. The process of finding a set of hyperparameters which can improve the performance of convolutional neural network is known as hyperparameter optimization, and it can be also called as tuning. In this Capstone project we are using genetic algorithm for finding an optimal CNN architecture. When the convolutional neural network goes deeper and more complicated, the number of hyperparameters increases dramatically. In order to increase the performance of tuning, we no longer use manual tuning methods like grid search,  instead we use genetic algorithm which is a kind of random search method for tuning, because we want the optimization algorithm searching automatically in a larger hyperparamter space.  To test the performance of genetic tuning, we implement genetic algorithm on MNIST which is a relative small data set include 60,000 examples in training set and 10,000 examples in test set. If this algorithm performs well in this dataset, we will have possibility to implement it on other significative datasets.
\pagebreak
\section{Introduction}

\section{Model \& Algorithm}
\subsection{CNN Model}

\subsection{Genetic Algorithm}

\section{Result}

\section{Conclusion}

\end{document}